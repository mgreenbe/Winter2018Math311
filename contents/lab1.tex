\documentclass[12pt]{amsart}

\usepackage{fullpage, amsmath, amsthm, amssymb, verbatim, graphicx}

\newcommand{\RR}{\mathbb{R}}
\newcommand{\ZZ}{\mathbb{Z}}
\DeclareMathOperator{\rref}{rref}
\DeclareMathOperator{\nullity}{nullity}
\DeclareMathOperator{\Span}{span}
\DeclareMathOperator{\rank}{rank}

\newtheorem{theorem}{Theorem}[section]
\newtheorem{thdf}{Theorem-Definition}[section]
\newtheorem{corollary}[theorem]{Corollary} \newtheorem{lemma}[theorem]{Lemma}
\theoremstyle{definition} \newtheorem{definition}[theorem]{Definition}
\newtheorem{provdef}[theorem]{Provisional definition}
\newtheorem{definitions}[theorem]{Definitions}
\newtheorem{remark}[theorem]{Remark} \newtheorem{remarks}[theorem]{Remarks}
\newtheorem{example}[theorem]{Example}
\newtheorem{exdef}[theorem]{Example-Definition}
\newtheorem{exercise}[theorem]{Exercise}

\newcommand{\ba}{\mathbf{a}}
\newcommand{\bb}{\mathbf{b}}
\newcommand{\bc}{\mathbf{c}}
\newcommand{\bd}{\mathbf{d}}
\newcommand{\be}{\mathbf{e}}
\newcommand{\bi}{\mathbf{i}}
\newcommand{\bu}{\mathbf{u}}
\newcommand{\bv}{\mathbf{v}}
\newcommand{\bw}{\mathbf{w}}
\newcommand{\bx}{\mathbf{x}}
\newcommand{\by}{\mathbf{y}}
\newcommand{\bz}{\mathbf{z}}
\newcommand{\bzero}{\mathbf{0}}

\newcommand{\bas}{\ba_1,\ldots,\ba_n}
\newcommand{\mat}[1]{\begin{bmatrix}#1\end{bmatrix}}
\newcommand{\Rmn}{\RR^{m\times n}} \newcommand{\Rmm}{\RR^{m\times m}}
\newcommand{\Rnn}{\RR^{n\times n}}
\newcommand{\spn}[1]{\Span\left(#1\right)}

\newcommand{\setstuff}{\setlength{\parskip}{0.5em}\setlength{\parindent}{0em}\setlength{\itemsep}{0.5em}}

\begin{document}
\title{MATH 311 -- Winter 2018 -- Lab 1}
\maketitle

\begin{enumerate}
  \setlength{\itemsep}{1em}
  \item Let
    \[
      U := \left\{\mat{x\\y\\z}\in\RR^3 : 21x-7z = 0\right\}.
    \]
    \begin{enumerate}
  \setlength{\itemsep}{0.5em}
      \item Find a matrix $A$ such that $U=N(A)$.
      \item By solving the system $A\bx=\bzero$, find a matrix $B$ such that $U=C(B)$.
    \end{enumerate}


  \item Let
    \[
      A = \mat{1 & 2 & 3\\4 & 5 & 6\\7 & 8 & 9},\quad
      \bb_1 = \mat{4\\-7\\-10},\quad
      \bb_2 = \mat{0\\0\\1}.
    \]
    \begin{enumerate}
  \setlength{\itemsep}{0.5em}
      \item Write $\bb_j$ as a linear combination of the columns of $A$, if possible.
      \item Prove that each column of $A$ is in the span of the other two.
    \end{enumerate}

  \item Let
    \[
      U := \left\{\mat{x\\y\\z}\in\RR^3 : x^2 + y^2 = z^2\right\}.
    \]
    \begin{enumerate}
  \setlength{\itemsep}{0.5em}
      \item Prove that $U$ contains $\bzero$.
      \item Prove that for all $\bu\in U$ and for all $t\in\RR$, $t\bu\in\RR$.
      \item Find vectors $\bu_1,\bu_2\in U$ such that $\bu_1+\bu_2\notin U$. Conclude that $U$ is not a subspace of $\RR^3$.
    \end{enumerate}
  \item Let $\bv\in\RR^n$ and let
    \[
      U := \{\bx\in\RR^n : \bv\cdot\bx = 0\}.
    \]
    \begin{enumerate}
  \setlength{\itemsep}{0.5em}
      \item Prove, from the definition, that $U$ is a subspace of $\RR^n$.
      \item Find a matrix $V\in \RR^{1\times n}$ such that $U=N(V)$. Conclude, again, that $U$ is a subspace of $\RR^n$.
    \end{enumerate}

    
  \item Let
    \[
      U = \left\{\mat{x + y\\x-y\\3x} : x,y\in\RR\right\}.
    \]
    \begin{enumerate}
  \setlength{\itemsep}{0.5em}
      \item Find a matrix $A\in\RR^{3\times 2}$ such that $U=C(A)$.
      \item Find a column vector $\bv\in\RR^3$ that is orthogonal to both columns of $A$.
      \item Show that
        \[
          \bv^TA = \mat{0&0}.
        \]
        (Hint: Interpret the columns of $\bv^TA$ as dot products.)
      \item Use (c) to show that
        \[
          \bv^TA\bx=\mat{0},
        \]
        for all $\bx\in\RR^2$. Conclude that $C(A)$ is a subset of $N(\bv^T)$.
      \item Interpret $N(\bv^T)$ as a plane in $\RR^3$, passing throgh the origin.
        Find an equation for it, in the form
        \[
          ax+by+cz=0
        \].
      \item By solving the system $\bv^T\bx=\bzero$, find a matrix $B\in\RR^{3\times 2}$ such that
        \[
          N(\bv^T)=C(B).
        \]
      \item ($*$) Prove that $C(A)=C(B)$.
    \end{enumerate}
\end{enumerate}
\end{document}
